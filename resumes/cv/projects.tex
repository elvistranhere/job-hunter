\cvsection{Projects}

\begin{cventries}

%---------------------------------------------------------

\cventry
    {C\#, .NET 8, SQL Server, React.js, TypeScript} % Technologies used
    {Awaire.ai} % Project name
    {Apr. 2025 - Dec. 2025} % Date(s)
    {Proprietary} % Location (left empty)
    {
      \begin{cvitems}
        \item {Served as Team Lead and full-stack engineer building a large-scale requirement-engineering platform for enterprise clients.}
        \item {Designed and implemented a hierarchical project-management system based on Afterburner’s FLEX methodology.}
        \item {Developed core business logic services in .NET, optimized SQL Server queries, and contributed to scalable system architecture decisions.}
        \item {Collaborated closely with Afterburner stakeholders to translate complex business processes into intuitive workflow UX.}
      \end{cvitems}
    }

%---------------------------------------------------------

\cventry
    {C++, Data Compression, Algorithms, Multithreading} % Technologies used
    {Maptek Geological Block Compression Project} % Project name
    {Aug. 2025 - Nov. 2025} % Date(s)
    {Proprietary} % Location (empty)
    {
      \begin{cvitems}
        \item {Engineered a high-performance compression algorithm for geological block models targeting extreme file-size reduction with minimal decompression error.}
        \item {Optimized memory usage and computation throughput, experimenting with quantization, spatial indexing, and multithreaded encoding pipelines.}
      \end{cvitems}
    }

%---------------------------------------------------------

\cventry
    {TypeScript, React, Capacitor, Zustand, iOS/macOS} % Technologies used
    {Moodist (Cross-Platform Port)} % Project name
    {Jan. 2026 - Present} % Date(s)
    {\href{https://github.com/remvze/moodist}{github.com/remvze/moodist \space\space \faLink}} % GitHub link
    {
      \begin{cvitems}
        \item {Collaborating with the creator of Moodist (3.2k+ GitHub stars) to port the web platform into native iOS, macOS, and iPadOS applications.}
        \item {Leveraging Capacitor to bridge the existing React codebase into high-performance native environments while maintaining a 1:1 design parity.}
        \item {Optimizing state management with Zustand and re-engineering mobile-first navigation to ensure a smooth, native-feel user experience.}
      \end{cvitems}
    }
%---------------------------------------------------------

\cventry
    {TypeScript, Next.js, React.js, MongoDB Atlas, Clerk, Tailwind CSS, Stripe, Cloudinary} % Technologies used
    {Fanta Create} % Project name
    {Sep. 2024 - Feb. 2025} % Date(s)
    {\href{https://fanta-create.vercel.app/}{fanta-create.vercel.app\space\space\faLink}} % Location
    {
      \begin{cvitems}
        \item {Developed an AI-powered platform for creators to generate images from prompts and utilize advanced editing tools like AI-assisted cropping and expansion.}
        \item {Implemented secure user authentication with Clerk, optimized data storage with MongoDB Atlas, and integrated Stripe for seamless payment processing.}
      \end{cvitems}
    }

%---------------------------------------------------------

\cventry
    {TypeScript, React, Vite, Express, Tailwind CSS, Shadcn/ui, Drizzle ORM, NeonDB, Passport.js, Recharts} % Technologies used
    {WealthTracker} % Project name
    {Mar. 2025 - May. 2025} % Date(s)
    {\href{https://github.com/totargaming/WealthTracker}{Repository\space\space\faGithub}} % Location
    {
      \begin{cvitems}
        \item {Developed a full-stack wealth tracking application featuring portfolio management and visualization.}
        \item {Implemented secure user authentication using Google OAuth 2.0 via Passport.js.}
        \item {Created interactive portfolio pages with data charts and CRUD operations for managing positions.}
        \item {Built an admin panel for user management and system configuration.}
      \end{cvitems}
    }

%---------------------------------------------------------

\end{cventries}
